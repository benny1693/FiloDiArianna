\section{Analisi delle caratteristiche utenti}
Il sito web si rivolge prevalentemente verso un pubblico di studenti e insegnanti delle scuole secondarie di secondo grado, ma anche verso un pubblico più variegato di persone che desiderano approfondire le proprie conoscenze della cultura greca.\\
Il target è potenzialmente molto ampio: la platea principale è quella degli studenti delle superiori che frequentano licei che prevedono l'insegnamento della cultura classica (qui in Italia si parla di almeno un milione studenti\footnote{\url{http://dati.istruzione.it/espscu/index.html?area=anagStu}}). Si aggiungono, inoltre, coloro i quali siano interessati alla cultura classica.

Il pubblico è composto principalmente da utenti che da sempre hanno navigato sul web e sanno come funziona. Data l'elevata quantità di utenza, il numero di persone non avvezze all'uso della tecnologia potrebbe rivelarsi lo stesso abbastanza corposo e rilevante.

\subsection{Studenti}
Gli studenti sono una tipologia di utente molto giovane e abituata al web moderno: rapido, agile e immediato. La loro primaria necessità è quella di effettuare ricerche in maniera veloce, molto probabilmente dal proprio dispositivo mobile, e di reperire l'informazione cercata.
Secondariamente, tenderanno a interagire gli uni con gli altri, magari discutendo sugli argomenti cercati.

\subsection{Appassionati}
Gli appassionati, invece, spenderanno più tempo nell'effettuare ricerche, magari cercando argomenti correlati a quello originariamente cercato. Costoro effettueranno ricerche ad ampio spettro, senza magari pensare ad un argomento preciso. Gli appassionati sono la categoria con il maggior numero di utenti con scarse competenze tecnologiche. Tuttavia, costoro saranno quelli che più volenterosi nel partecipare alla discussione e alla condivisione delle proprie conoscenze. I dispositivi utilizzati da questa categoria di utenti possono appartenere a diverse tipologie, mobile o desktop. 

\subsection{Conclusioni}

\subsubsection{Ricerca}
Il sito dovrà dare la possibilità di cercare argomenti sia in maniera precisa, sia tramite ricerche ad ampio spettro. Gli strumenti da utilizzare potrebbero essere una barra di ricerca per coloro i quali cercano un argomento particolare e una sezione del sito che, invece, permette l'esplorazione delle voci contenute tramite raggruppamenti in categorie. Il sito dovrà essere responsive, in modo da potersi adattare a ogni tipo di dispositivo, e dovrà essere semplice e intuitivo, sia per favorire gli utenti meno avvezzi alla tecnologia, sia per rendere rapida la navigazione per gli utenti più esperti.

\subsubsection{Interazioni}
\paragraph{Inserimento, modifica e rimozione}
Il sito dovrà dare la possibilità agli utenti di inserire nuove voci, modificarle ed eventualmente eliminarle per venire incontro alle esigenze degli appassionati che intenderanno partecipare più attivamente alla vita della piattaforma. Inoltre, deve essere disponibile la possibilità di mettere in relazione i vari argomenti tra di loro.

\paragraph{Social}
Il sito dovrà dare la possibilità agli utenti di interagire fra di loro tramite strumenti di comunicazione. Gli utenti devono avere dunque la possibilità di creare un proprio profilo, possibilmente visualizzabile da chiunque altro.