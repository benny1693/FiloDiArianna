\section{Fase di Progettazione}

\subsection{Struttura del sito}
Il sito presenta tre livelli di profondità in modo tale da rendere facile la ricerca manuale senza dover rinunciare ad una organizzazione gerarchica ben strutturata.\\
La struttura della gerarchia del sito è suddivisa in categorie, sottocategorie e infine pagine, ed è così definita:
\begin{itemize}
	\item \textbf{Personaggi:}
	\begin{itemize}
		\item Esseri umani;
		\item Semidivinità/eroi;
		\item Divinità;
		\item Creature.
	\end{itemize}
	\item \textbf{Eventi:}
	\begin{itemize}
		\item Epoca degli dei;
		\item Epoca degli dei e degli uomini;
		\item Epoca degli eroi.
	\end{itemize}
	\item \textbf{Luoghi:}
	\begin{itemize}
		\item Mitologici;
		\item Reali.
	\end{itemize}
\end{itemize}
Le pagine possono essere in due stati differenti:
\begin{itemize}
	\item \textbf{Pendenti:} sono pagine in attesa di approvazione da parte di un amministratore per poter essere pubblicate, non sono visibili sul sito web;
	\item \textbf{Pubblicate:} sono pagine visibili nel sito web.
\end{itemize}

\subsubsection{Header}
All'interno dell'header è presente il \textbf{logo} del sito, la \textbf{barra di ricerca} e due \textbf{pulsanti} che vengono visualizzati in base al tipo di utente (registrato, non registrato) e il \textbf{menu}.\\
Il \textbf{logo} è situato a sinistra ed è privo di link, infatti il link che rimanda alla homepage è già presente all'interno del menu nella voce Home.\\
La \textbf{barra di ricerca} è situata a destra del logo ed orizzontalmente centrato rispetto alla pagina. Questo perché essendo un sito che viene usato principalmente per la ricerca di informazioni riguardanti la mitologia greca, l'utente deve avere subito ciò che vuole, ovvero poter cercare ciò di cui ha bisogno e di fatti lo trova immediatamente. Grazie al menu a tendina situato alla sinistra del campo di ricerca, l'utente può selezionare la categoria all'interno della quale fare la propria ricerca.\\
Esistono due tipi di \textbf{pulsanti} situati a destra della barra di ricerca:
\begin{itemize}
	\item \textbf{Accedi:} viene visualizzato se l'utente non ha effettuato il login al sito;
	\item \textbf{Area Riservata:} viene visualizzato se l'utente ha effettuato il login a sito.
\end{itemize}
Il \textbf{menu} viene visualizzato come left sidebar, ma fa parte dell'header. Abbiamo scelto questa visualizzazione perché essendo stato misurato il movimento oculare fatto dagli utenti in una pagina web, da cui si è ricavata una termografia dove le zone calde rappresentano le parti della pagina in cui l'utente ha dedicato più tempo, le quali risultano avere una forma ad F. Di conseguenza avendo il menu a sinistra, l'attenzione ricade su di esso molto più facilmente.

\subsection{Accessibilità}