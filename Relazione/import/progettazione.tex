\section{Fase di Progettazione}
Come strategie di progettazione abbiamo usato quella del \textbf{Mobile First} e \textbf{Responsive Web Design}.\\
\textbf{Mobile First} permette di caricare gli elementi essenziali sulle piattaforme mobile. Ciò porta un'esperienza di navigazione più snella, che evita ritardi e latenza di caricamento.\\
\textbf{Responsive Web design} si basa sul concetto di media query che ottimizzano la visualizzazione in base a dispositivi specifici e dimensioni di viewport.\\
Infatti è stato codificato il CSS iniziale secondo un approccio \textbf{Mobile First} e poi utilizzato le media query per servire selettivamente contenuti aggiuntivi al crescere delle dimensioni del viewport.

\subsection{Struttura del sito}
Il sito presenta tre livelli di profondità in modo tale da rendere facile la ricerca manuale senza dover rinunciare ad una organizzazione gerarchica ben strutturata.\\
La struttura della gerarchia del sito è suddivisa in categorie, sottocategorie e infine pagine, ed è così definita:
\begin{itemize}
	\item \textbf{Personaggi:}
	\begin{itemize}
		\item Esseri umani;
		\item Semidivinità/eroi;
		\item Divinità;
		\item Creature.
	\end{itemize}
	\item \textbf{Eventi:}
	\begin{itemize}
		\item Epoca degli dei;
		\item Epoca degli dei e degli uomini;
		\item Epoca degli eroi.
	\end{itemize}
	\item \textbf{Luoghi:}
	\begin{itemize}
		\item Mitologici;
		\item Reali.
	\end{itemize}
\end{itemize}
Le pagine possono essere in due stati differenti:
\begin{itemize}
	\item \textbf{Pendenti:} sono pagine in attesa di approvazione da parte di un amministratore per poter essere pubblicate, non sono visibili sul sito web;
	\item \textbf{Pubblicate:} sono pagine visibili nel sito web.
\end{itemize}

\subsubsection{Header}
All'interno dell'header è presente il \textbf{logo} del sito, la \textbf{barra di ricerca} e due \textbf{pulsanti} che vengono visualizzati in base al tipo di utente (registrato, non registrato) e il \textbf{menu}.\\
Il \textbf{logo} è situato a sinistra ed è privo di link, infatti il link che rimanda alla homepage è già presente all'interno del menu nella voce Home.\\
La \textbf{barra di ricerca} è situata a destra del logo ed orizzontalmente centrato rispetto alla pagina. Questo perché essendo un sito che viene usato principalmente per la ricerca di informazioni riguardanti la mitologia greca, l'utente deve avere subito ciò che vuole, ovvero poter cercare ciò di cui ha bisogno. Grazie al menu a tendina situato alla sinistra del campo di ricerca, l'utente può selezionare la categoria all'interno della quale fare la propria ricerca.\\
Per quanto riguarda la quantità del testo visibile che un utente può scrivere all'interno della barra di ricerca, è stato scelto un numero pari a 30 caratteri. Questo perché un box troppo piccolo aumenta lo stress proporzionalmente per ogni carattere che sfora e gli utenti tendono a scrivere meno con risultati della query più scarsi.\\
Esistono due tipi di \textbf{pulsanti} situati a destra della barra di ricerca:
\begin{itemize}
	\item \textbf{Accedi:} viene visualizzato se l'utente non ha effettuato il login al sito;
	\item \textbf{Area Riservata:} viene visualizzato se l'utente ha effettuato il login a sito.
\end{itemize}
Il posizionamento di questi due pulsanti è la stessa anche per la visualizzazione da smartphone, infatti compaiono sulla stessa riga del logo posizionati a destra.\\
Il \textbf{menu} viene compare come left sidebar nella visualizzazione per Desktop, ma fa parte dell'header. Abbiamo scelto questo posizionamento perché, dagli studi fatti sul movimento oculare sugli utenti in una pagina web, si è ricavata una termografia dove le zone calde rappresentano le parti della pagina in cui l'utente ha dedicato più tempo, le quali risultano avere una forma ad F. Di conseguenza avendo il menu a sinistra, l'attenzione ricade su di esso in modo spontaneo.\\
Per quanto riguarda la visualizzazione per smartphone, il menu è nascosto al di fuori dell'area visibile del sito. Per farlo comparire basta premere il burger menu in alto a destra. Abbiamo scelto quel posizionamento perché il 49\% degli utenti usa lo smartphone ad una mano e il pollice come puntatore. Siccome la maggior parte di essi non è mancina, riusciranno a raggiungere più facilmente il pulsante. Per quanto riguarda l'altra percentuale di utenti, non avranno problemi al raggiungimento del pulsante perché usano lo smartphone a due mani.\\
Per questi motivi il menu verrà visualizzato coprendo l'intera larghezza dello schermo con le voci al centro.

\subsubsection{Footer}
All'interno del footer compare solo la dicitura per il copyright. Potevamo aggiungere i contatti ma, essendo sempre visibile la voce del menu che porta alla pagina contatti, non aveva senso duplicare le informazioni.

\subsubsection{Breadcrumb}
Il breadcrumb è presente in tutte le pagine del sito, sia mobile che desktop, e comprende un insieme di campi che identificano la posizione dell'utente all'interno del sito. L'ultimo campo è la pagina corrente, ovvero la pagina che l'utente sta visualizzando. Per evitare i link circolari, quest'ultimo campo è solo un testo.

\subsubsection{Contenuto}
Il sito ha lo scopo principale di ricerca e lettura delle informazioni e per questo motivo abbiamo scelto un layout stretto. Infatti, cosi facendo, il numero medio di parole scritte in una riga è di circa 15 e questo permette una migliore lettura.\\
Le informazioni più importanti compaiono nella prima metà della pagina siccome è quella più visibile e che l'utente vede appena viene caricato il sito. Inoltre abbiamo cercato di mettere anche gli elementi fondamentali per l'interazione, come il menu e la barra di ricerca.\\
Lo schema che abbiamo utilizzato è quello a tre panelli, che rispondono alle seguenti domande:\\
\begin{itemize}
	\item \textbf{Dove sono?:} la prima risposta a questa domanda si trova nel titolo della pagina. Sono brevi e vanno dal particolare al generale. La seconda si trova nei breadcrumb, che indicano il percorso fatto per arrivare in quel punto;
	\item \textbf{Di cosa si tratta?:} la risposta a questa domanda si trova nel contenuto della pagina;
	\item \textbf{Dove posso andare?:} la risposta a questa domanda si trova nel menu che compare come left sidebar.
\end{itemize}

\subsubsection{Database}


\subsection{Accessibilità}

\subsubsection{Separazione tra contenuto, presentazione e struttura}
Per migliorare l’accesso al sito agli utenti con differenti disabilità e ai motori di ricerca è stata mantenuta la separazione tra struttura, presentazione e comportamento.\\
La prima è stata sviluppata tramite documenti XHTML5, i quali richiamano i fogli di stile esterni CSS che implementano la presentazione e script esterni realizzati con JavaScript che formano il comportamento. Questi script sono stati implementati in modo da garantire una trasformazione elegante del sito, poiché se JavaScript è disabilitato il contenuto rimane comunque accessibile.\\
Tutto il codice redatto è stato scritto secondo le raccomandazioni W3C, accertando che fossero state rispettate tramite validazione. Si è evitato l’uso di tag e attributi deprecati.