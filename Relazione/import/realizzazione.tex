\section{Fase di Realizzazione}
Di seguito si elencano le parti realizzate da ciascun componente del gruppo:
\begin{itemize}
	\item \textbf{Cameran Laura:}
	\begin{enumerate}
		\item Creazione classi PHP;
		\item CSS di alcune parti del sito;
		\item Alcune funzioni JavaScript;
		\item HTML di alcune pagine.
	\end{enumerate}
	\item \textbf{Cosentino Benedetto:}
	\begin{enumerate}
		\item Creazione Database e procedure;
		\item Creazione classi PHP;
		\item CSS di alcune parti del sito;
		\item HTML di alcune pagine;
		\item Stesura della relazione.
	\end{enumerate}
	\item \textbf{Ranzato Matteo:}
	\begin{enumerate}
		\item Realizzazione pagina index;
		\item File JavaScript;
		\item CSS per mobile, tablet e desktop;
		\item HTML di alcune pagine;
		\item Stesura della relazione.
	\end{enumerate}
	\item \textbf{Spreafico Alessandro:}
	\begin{enumerate}
		\item Creazione classi PHP;
		\item CSS di alcune parti del sito;
		\item HTML di alcune pagine;
		\item Stesura della relazione.
	\end{enumerate}
\end{itemize}
Abbiamo iniziato a lavorare creando le pagine in HTML, definendo prima l'header e il footer insieme e poi inserendo i contenuti separatamente. Successivamente abbiamo iniziato a lavorare sul CSS definendo le parti di ognuno evitando la creazione di stesse classi, infatti abbiamo definito insieme le parti che erano in comune con tutti e poi si è lavorato in parallelo su file separati. Per vedere se tutto andasse bene, nell'HTML sono stati importati tutti i file e, una volta che tutte le pagine sono state approvate, abbiamo portato tutti i CSS in un unico file e le successive piccole modifiche sono state apportate direttamente su quest'ultimo file.\\
Abbiamo poi lavorato parallelamente al Javascript che al PHP, ma anche al Database. Con JavaScript abbiamo usato la programmazione procedurale cercando di creare codice più modulare possibile. Con il PHP invece, siamo partiti definendo tramite UML una gerarchia e, una volta approvata, abbiamo creato le varie classi e le abbiamo testate. Fatto questo, le pagine HTML sono state convertite in pagine PHP. Alla fine abbiamo testato tutte le parti del sito per verificare che non ci fossero errori nel codice e nel comportamento del sito. Una volta verificato ciò, i file JavaScript e CSS sono stati minimizzati per ridurre il peso complessivo del sito.