\section*{Abstract}

Il filo di Arianna è stato sviluppato per fornire all'omonima associazione di promozione culturale un mezzo moderno per la diffusione della cultura dell'antica Grecia.
Il nome scelto è una metafora che fa riferimento al mito di Teseo e Arianna.
Nel mito, Teseo è in grado di non perdersi dentro il labirinto grazie al gomitolo di lana che Arianna gli dona. Teseo, infatti, mentre procede srotola il gomitolo, così da poterlo seguire a ritroso e trovare l'uscita del labirinto dopo aver ucciso il Minotauro.
Come il filo permise a Teseo di non perdersi nel labirinto, così il sito web dell'associazione permette permette agli utenti di non perdersi nella ricerca della conoscenza della cultura greca.

Sul sito è disponibile una raccolta di pagine riguardanti la cultura greca. 
Le pagine sono divise in varie categorie di cui le principali sono Personaggi, Eventi e Luoghi. 

Il sito permette di inserire, modificare ed eventualmente eliminare gli articoli pubblicati. Infine, è possibile confrontarsi con gli altri utenti riguardo agli argomenti trattati tramite un'area discussione messa a disposizione per ogni articolo pubblicato.